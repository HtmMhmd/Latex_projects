\documentclass[14pt,a4paper]{report}
\usepackage{tocloft}
\usepackage{hyperref}
\usepackage{lastpage}
\usepackage{xcolor}
\usepackage{amsmath}
\usepackage{fancyhdr}
\usepackage{enumitem}
\usepackage{tabularx}
\usepackage[skip=2pt,font=small]{caption}
\usepackage{environ}
\usepackage{mdframed}
\usepackage{graphicx}
\usepackage{ntheorem}


% Booleans to show answers and to ask students to return the question form
\newif\ifshowanswers

% ==============================================================================
% Question command
%
\newcounter{question}
\makeatletter
\@addtoreset{question}{section} % reset question counter at every section
\makeatother
\newcommand*\question{%
\stepcounter{question}%
\paragraph{Question \thesection.\thequestion}}

% ==============================================================================
% AnsweR_boxes
%
\mdtheorem[outerlinewidth=2,roundcorner=10pt,
leftmargin=0,rightmargin=0,
backgroundcolor=yellow!40,outerlinecolor=blue!70!black,
innertopmargin=\topskip,splittopskip=\topskip,
ntheorem=true,]{answer_box}{Answer}[section]

\NewEnviron{answer}{
\ifshowanswers
\begin{answer_box*}
\BODY
\end{answer_box*}
\fi}

% ==============================================================================
% Show answers?
%
\showanswerstrue
% \showanswersfalse


% ==============================================================================
%
% Course information
%
% ==============================================================================

\newcommand{\coursename}{Introduction to Digital Communication}
\newcommand{\coursecode}{ECE 382}

\newcommand{\assigntype}{Portfolio}
\newcommand{\teacher}{Hatem Maohamed Ahmed (\texttt{20010447})}



% ==============================================================================
%
% Margins, header and footer
%
% ==============================================================================
\setlength{\topmargin}{0cm}
\setlength{\textheight}{9.25in}
\setlength{\oddsidemargin}{0.0in}
\setlength{\evensidemargin}{0.0in}
\setlength{\textwidth}{16cm}
\pagestyle{fancy}
\cfoot{\footnotesize{Page \thepage \ of \pageref{finalpage}
-- \chaptername \ \thechapter \ \sectionautorefname \ \thesection}}
\renewcommand{\headrulewidth}{0pt}
\renewcommand{\footrulewidth}{0pt}

\begin{document}

% ==============================================================================
%
% Header
%
% ==============================================================================

\noindent\makebox[\linewidth]{\rule{\textwidth}{0.4pt}}
\begin{center}
\Large \textbf{\coursename} (\coursecode)
\end{center}

\begin{center}
\large \assigntype{} \\
\vspace{3mm}
\end{center}

\begin{center}
\teacher\\

\end{center}

\noindent\makebox[\linewidth]{\rule{\textwidth}{0.4pt}}



% Portfolio Content
\tableofcontents

% ==============================================================================
%
% Chapter 1
%
% ==============================================================================


\chapter{Transmission through band limited channels}
\noindent\makebox[\linewidth]{\rule{\textwidth}{0.4pt}}

% --------------------------------------------------------------------------------

% ==========================begin Question of Note 1==========================
\section{Note 1}
\question 

{
    What are the bandwidths for different communication channels?
}

\begin{answer_box*}
{
\begin{tabular}{|l|l|}
    \hline
    Communication Channel & Bandwidth \\
    \hline
    \textbullet\ Telephone channels & \textbullet\ 4 kHz \\
    \textbullet\ Microwave LOS radio channel & \textbullet\ 40 MHz (2.4 GHz) to 2.16 GHz (60 GHz) \\
    \textbullet\ Satellite channel & \textbullet\ 36 MHz (C-band) to 500 MHz (Ka-band) \\
    \textbullet\ Underwater acoustic channel & \textbullet\ A few tens of Hz to a few kHz \\
    \hline
\end{tabular}
}

\paragraph{Telephone channels:}

The bandwidth of a telephone channel is typically 4 kHz. This bandwidth is sufficient to transmit voice signals with good quality, but not enough for transmitting high-quality audio or video.

\paragraph{Microwave LOS radio channel:}

The bandwidth of microwave LOS radio channels can vary depending on the frequency band used. For example, the bandwidth of a 2.4 GHz microwave link can be up to 40 MHz, while a 60 GHz microwave link can have a bandwidth of up to 2.16 GHz. Microwave LOS radio channels are often used for high-speed data transmission, such as wireless broadband internet.

\paragraph{Satellite channel:}

The bandwidth of a satellite channel also varies depending on the type of satellite and the frequency band used. For example, a typical satellite channel in the C-band has a bandwidth of around 36 MHz, while a Ka-band satellite channel can have a bandwidth of up to 500 MHz. Satellite channels are often used for long-distance communication, such as satellite TV and satellite internet.

\paragraph{Underwater acoustic channel:}

The bandwidth of an underwater acoustic channel is much lower than that of other communication channels due to the high attenuation of sound waves in water. The bandwidth of an underwater acoustic channel can range from a few tens of Hz to a few kHz, depending on the distance between the transmitter and the receiver, and the frequency used. Underwater acoustic channels are often used for underwater communication, such as underwater sonar and underwater modems.

\end{answer_box*}

\question
{
    Why is a wireless channel time-varying?
}
\begin{answer_box*}

    Wireless channels are time-varying because the electromagnetic waves that carry the wireless signals can experience a variety of changes as they propagate through the environment. These changes can occur due to several factor expressed below: \\

    \begin{tabular}{|l|p{0.7\linewidth}|p{0.3\linewidth}|}
            \hline
            Reflection & When a wireless signal encounters an obstacle, it can reflect off the surface and reach the receiver via a longer path, leading to time-varying signal interference. \\
            \hline
            Diffraction & When a wireless signal encounters a large obstacle, it can bend around it and reach the receiver, causing signal fading due to constructive and destructive interference. \\
            \hline
            Scattering & When a wireless signal encounters small objects, it can scatter in multiple directions, resulting in time-varying signals and multipath interference. \\
            \hline
            Doppler effect & When the transmitter or receiver is in motion, the frequency of the received signal can shift due to the Doppler effect, causing time-varying signal strength. \\
            \hline
    \end{tabular}

\end{answer_box*}

\question
{
    Using MATLAB, generate signal $x(t) = sinc(t/0.01)$ , $H c (f ) = e ^{( - j * 2 * \pi * (f + 10f 2 ))}$ plot output if $x(t)$ passes through $H c (f )$ ?
}
\begin{answer_box*}
    \begin{verbatim}
    % Define the sampling frequency
    fs = 1000;

    % Define the time vector
    t = -1:1/fs:1;

    % Define the signal x(t)
    x = sinc(t/0.01);

    % Define the frequency vector
    f = -fs/2:fs/length(x):fs/2-fs/length(x);

    % Define the channel frequency response Hc(f)
    Hc = exp(-1j * 2 * pi * (f + 10 * f.^2));

    % Compute the output of the channel
    y = x .* Hc;

    % Plot the input and output signals
    figure;
    subplot(2,1,1);
    plot(t, x);
    title('Input Signal x(t)');
    xlabel('Time (t)');
    ylabel('Amplitude');

    subplot(2,1,2);
    plot(f, abs(fftshift(fft(y))));
    title('Output Signal Y(f)');
    xlabel('Frequency (f)');
    ylabel('Magnitude');
    \end{verbatim}
    
\end{answer_box*}

\question
{
    Do we need to have zero ISI at each time instant? or just at sampling time?
}
\begin{answer_box*}
    No, it is only necessary to have zero ISI at the sampling time.
\end{answer_box*}

\question{
    Why \(g_T(t) \ast h_c(t)\)?
}
\begin{answer_box*}
    Because the received signal is distorted by the channel response \(h_c(t)\) as it propagates through the communication channel. This distortion can cause the transmitted pulses to spread in time and overlap with each other, leading to inter-symbol interference (ISI). By designing the receiver filter to be matched to \(g_T(t) \ast h_c(t)\), we can effectively remove the ISI and improve the detection of the transmitted symbols.
\end{answer_box*}

\question{
    If \(a_k \in \{1, -1\}\), what should be \(V_{th}\)?
}
\begin{answer_box*}
    In a binary communication system with binary antipodal signaling, the threshold voltage \(V_{th}\) should be set to 0. This ensures that the decision boundary is midway between the two possible transmitted signal values (+1 and -1).
\end{answer_box*}

\question{
Why do we need to have ISI = 0 only at the sampling time?
}
\begin{answer_box*}
    ISI must be zero at the sampling time to ensure accurate symbol detection. Equalization techniques adjust the received signal to minimize ISI at this critical moment.
\end{answer_box*}

\question
{
    Why is optimal sampling at the maximum eye opening?
}
\begin{answer_box*}
    Optimal sampling occurs at the maximum eye opening because:\\
    • Symbols are most separated, reducing ISI effects. \\ 
    • Signal-to-noise ratio (SNR) is highest, improving receiver performance and reducing symbol errors.
\end{answer_box*}

\question
{
    Sketch the eye pattern in case the bandwidth of the channel is $\infty$.
}
\begin{answer_box*}

    With infinite bandwidth, ISI is eliminated, resulting in a wide-open eye pattern with distinct horizontal lines and vertical transitions representing symbols.

\end{answer_box*}

\question
{
    Why outside sampling times, overlapping pulses is of no practical significance?
}
\begin{answer_box*}
    Outside the sampling times, overlapping pulses do not contribute to the received signal because the sampler only takes samples at specific time intervals. The only concern is the pulse shape at the sampling times, which determines the distortion due to ISI.
\end{answer_box*}

\question 
{
    Why filters need to be causal?
}
\begin{answer_box*}
    Causality ensures that the filter has a physical realizable implementation. A causal filter is one whose output depends only on present and past input values. This means that the filter can be implemented in real-time, without requiring a delay line or other mechanism to store future values of the input signal.
\end{answer_box*}

\question 
{
    Why it is required to have aliasing between replicas of $P(f)$?
}
\begin{answer_box*}
    Aliasing between replicas of $P(f)$ is required to satisfy the Nyquist criterion and have constant amplitude = Tb by overlapping.\\
    The Nyquist criterion states that the sampling rate must be at least twice the highest frequency component of the signal to avoid aliasing.\\
    By overlapping the replicas of $P(f)$, we can ensure that the sampled signal has a constant amplitude. This is important for maintaining signal integrity and minimizing distortion.
\end{answer_box*}

\question 
{
    Verify that $P (nT_b ) = \delta (n)$ for $sinc(t/T_b )$?
}
\begin{answer_box*}
Proof:

We begin by taking the Fourier transform of the sinc function:
\begin{equation}
\left(\mathcal{F}[\text{sinc}(t/T_b)]\right) = \int_{-\infty}^{\infty} \frac{\sin(t/T_b)}{T_b} e^{-j2\pi ft} dt
\end{equation}
We then use the Fourier series expansion of the exponential function:
\begin{equation}
e^{-j2\pi ft} = \sum_{k=-\infty}^{\infty} \frac{(-1)^k}{j2\pi k} e^{-j2\pi kt}
\end{equation}
Substituting the Fourier series expansion into the integral, we get:
\begin{equation}
\mathcal{F}[\text{sinc}(t/T_b)] = \sum_{k=-\infty}^{\infty} \frac{(-1)^k}{j2\pi k} \int_{-\infty}^{\infty} \frac{\sin(t/T_b)}{T_b} e^{-j2\pi (k+f)t} dt
\end{equation}
Interchanging the order of integration and summation, we have:
\begin{equation}
\mathcal{F}[\text{sinc}(t/T_b)] = \sum_{k=-\infty}^{\infty} \frac{(-1)^k}{j2\pi k} \int_{-\infty}^{\infty} \frac{\sin(t/T_b)}{T_b} e^{-j2\pi (k+f)t} dt
\end{equation}
Evaluating the integral using the sifting property of the sinc function, we have:
\begin{equation}
\mathcal{F}[\text{sinc}(t/T_b)] = \sum_{k=-\infty}^{\infty} \frac{(-1)^k}{j2\pi k} \delta(k+f/T_b)
\end{equation}
Substituting $f = -k/T_b$ into the sum, we obtain:
\begin{equation}
\mathcal{F}[\text{sinc}(t/T_b)] = \sum_{k=-\infty}^{\infty} \frac{(-1)^k}{j2\pi k} \delta(k-f/T_b)
\end{equation}
Taking the inverse Fourier transform of both sides, we have:
\begin{equation}
\text{sinc}(t/T_b) = \sum_{k=-\infty}^{\infty} \frac{(-1)^k}{j2\pi k} \delta(k-f/T_b)
\end{equation}
Evaluating the inverse Fourier transform at $t = nT_b$, we obtain:
\begin{equation}
\text{sinc}(nT_b) = \sum_{k=-\infty}^{\infty} \frac{(-1)^k}{j2\pi k} \delta(k-n)
\end{equation}
Simplifying the sum, we have:
\begin{equation}
\text{sinc}(nT_b) = \frac{(-1)^n}{j2\pi n}
\end{equation}
Comparing the result to the definition of the Dirac delta function, we can see that:
\begin{equation}
P(nT_b) = \delta(n) \quad \text{for} \quad \text{sinc}(t/T_b)
\end{equation}
\end{answer_box*}

\question 
{
    Calculate min BW needed for telephone system using PCM with sampling rate of 8 kHz \& 256 quantization level.
}
\begin{answer_box*}
    To calculate the minimum bandwidth required for a PCM system with a sampling rate of $f_s = 8 \text{ kHz}$ and $N = 256$ quantization levels, we can use the formula:

    \begin{equation}
    BW = 2B = 2 \cdot (1 + \frac{79688}{10^6}) \cdot R
    \end{equation}
    
    where $B$ is the bandwidth, $\alpha$ is the excess bandwidth factor, and $R$ is the bit rate.
    
    The bit rate for a PCM system can be calculated as:
    
    \begin{equation}
    R = \log_2(N) \cdot f_s
    \end{equation}
    
    Substituting the given values, we get:
    
    \begin{equation}
    R = \log_2(256) \cdot 8 \text{ kHz} = 16 \text{ kbps}
    \end{equation}
    
    Therefore, the minimum bandwidth required for the telephone system using PCM is:
    
    \begin{equation}
    BW = 2B = 2 \cdot (1 + \frac{79688}{10^6}) \cdot 16 \text{ kbps} = 35200 \text{ Hz}
    \end{equation}
    
    So the minimum bandwidth needed is $35200 \text{ Hz}$.

\end{answer_box*}

\question
{
    find $t_0$ such that $sinc(2w(t - t_0 ))$ $<$ 0.001 for t $<$ 0
}
\begin{answer_box*}

\begin{align}
    \textbf{Case 1: t $<$ $t_0$}
\end{align}
    In this case, sinc(2w(t - $t_0$)) is given by

    $$sinc(2w(t - t_0)) = \frac{sin(2w(t - t_0))}{2w(t - t_0)}$$

    Since sin(x) $<$= x for all x, we have

    $$\frac{|sin(2w(t - t_0))|}{2w(t - t_0)} <= \frac{2w(t - t_0)}{2w(t - t_0)} = 1$$

    Therefore,

    $$|sinc(2w(t - t_0))| <= 1$$

    for all t $<$ $t_0$.

\begin{align}
    \textbf{Case 2: t = $t_0$}
\end{align}
    In this case, sinc(2w(t - $t_0$)) is given by

    $$sinc(2w(t - t_0)) = \frac{sin(2w(t - t_0))}{2w(t - t_0)} = \frac{sin(0)}{0} = \text{undefined}$$

    However, we can define sinc(0) to be 1, so we have

    $$sinc(2w(t - t_0)) = 1$$

    for t = $t_0$.

    Therefore, we have

    $$|sinc(2w(t - t_0))| <= 1$$

    for all t.
    
    To ensure that |sinc(2w(t - $t_0$))| $<$ 0.001 for t $<$ 0, we need to choose $t_0$ such that

    $$t_0 > \frac{\pi}{2w}$$

    This is because |sinc(x)| is maximized at x = 0, and |sinc(x)| $<$ 1 for all other x.

    Therefore, the smallest possible value of $t_0$ that satisfies both conditions is

    $$t_0 = \frac{\pi}{2w}$$
\end{answer_box*}

% ========================== end Question of Note 1=============================
% --------------------------------------------------------------------------------


% --------------------------------------------------------------------------------
% ==========================begin Question of Note 2==============================
\section{Note 2}
\question
{
    $P_RC (f + R_b ) + P_RC (f ) + P_RC (f - R_b ) = T_b \forall f \subset -R_b /2, R_b /2$
}

\begin{answer_box*}
    
Verification:

Step 1: Simplify the left-hand side

Using the definition of the raised cosine function, we can write each term on the left-hand side as a sum of two sinc functions:

\begin{equation}
    P_RC (f + R_b ) = \frac{1}{2}\left[\text{sinc}\left(\frac{f + R_b }{2Rc}\right) + \text{sinc}\left(\frac{f + R_b }{2Rc} - 1\right)\right]
\end{equation}

\begin{equation}
    P_RC (f ) = \frac{1}{2}\left[\text{sinc}\left(\frac{f }{2Rc}\right) + \text{sinc}\left(\frac{f }{2Rc} - 1\right)\right]
\end{equation}


\begin{equation}
    P_RC (f - R_b ) = \frac{1}{2}\left[\text{sinc}\left(\frac{f - R_b }{2Rc}\right) + \text{sinc}\left(\frac{f - R_b }{2Rc} - 1\right)\right]
\end{equation}


Substituting these expressions into the left-hand side, we get:

        $$P_RC (f +R_b ) + P_RC (f ) + P_RC (f -R_b ) =$$
        $$\frac{1}{2}\text{sinc}\left(\frac{f +R_b }{2Rc}\right) + \text{sinc}\left(\frac{f +R_b }{2Rc} - 1\right) + $$ 
        $$\text{sinc}\left(\frac{f }{2Rc}\right) + \text{sinc}\left(\frac{f }{2Rc} - 1\right) + $$
        $$\text{sinc}\left(\frac{f - R_b }{2Rc}\right) + \text{sinc}\left(\frac{f - R_b }{2Rc} - 1\right)    $$

Simplifying further, we get:

\begin{equation}
    P_RC (f +R_b ) + P_RC (f ) + P_RC (f -R_b ) = 
    \text{sinc}\left(\frac{f }{2Rc}\right) + \left[\text{sinc}\left(\frac{f +R_b }{2Rc}\right) + \text{sinc}\left(\frac{f -R_b }{2Rc}\right)\right]
\end{equation}


Step 2: Substitute into the original equation

Substituting the simplified expression for the left-hand side into the original equation, we get:

    $$\left[\text{sinc}\left(\frac{f }{2Rc}\right) + \left[\text{sinc}\left(\frac{f +R_b }{2Rc}\right) + \text{sinc}\left(\frac{f -R_b }{2Rc}\right)\right]\right] $$    
    $$\left[\text{sinc}\left(\frac{f +R_b }{2Rc}\right) + \text{sinc}\left(\frac{f -R_b }{2Rc}\right)\right] + $$
    $$\left[\text{sinc}\left(\frac{f + R_b }{2Rc}\right) + \text{sinc}\left(\frac{f - R_b }{2*Rc}\right) - T_b \right] = 0$$

Step 3: Simplify

Simplifying further, we get:
\begin{equation}
    \text{sinc}\left(\frac{f }{2Rc}\right) + \text{sinc}\left(\frac{f + R_b }{2Rc}\right) + \text{sinc}\left(\frac{f - R_b }{2*Rc}\right) = \frac{T_b }{2}
\end{equation}

Step 4: Check for all f in the range $-R_b /2, R_b /2$

We need to check if this equation holds for all f in the range $-R_b /2, R_b /2$, We can start by considering the case when $f = 0:$

\begin{equation}
    \text{sinc}(0) + \text{sinc}\left(\frac{R_b }{2Rc}\right) + \text{sinc}\left(-\frac{R_b }{2Rc}\right) = \frac{T_b }{2}
\end{equation}


Since the sinc function is symmetric around 0, we have sinc(-x) = sinc(x), which simplifies the equation to:
\begin{equation}
    2\text{sinc}(0) + \text{sinc}\left(\frac{R_b }{2Rc}\right) = \frac{T_b }{2}
\end{equation}

Since sinc(0) = 1, we can simplify further to:

\begin{equation}
    \text{sinc}\left(\frac{R_b }{2Rc}\right) = \frac{T_b }{2} - 2
\end{equation}

Now we need to check if this equation holds for all values of $R_b /(2Rc)$ in the range [-1, 1]. We can see that if $R_b /(2Rc)$ = 1, then the left-hand side of the equation becomes infinite, which means the equation does not hold. Therefore, the given condition does not hold for all f in the range $-R_b /2, R_b /2$.

\end{answer_box*}

\question
{
    $P_RC (f )$ exhibits odd symmetry with respect to $f = \frac{1}{2T_{b}}$. Verify odd symmetry.
}
\begin{answer_box*}

    Verification:

    To verify odd symmetry, we need to check whether the function satisfies the following property:
    \begin{equation}
        P_{RC}(f) = -P_{RC}(-f) \label{eq:odd_symmetry_condition}
    \end{equation}
    By Substituting in both:

    \begin{equation}
        P_{RC}(f) =  \frac{1}{T_b} \cdot \frac{\cos^2(-\pi f T_b)}{1 - \left({2\alpha (f)}{T_b}\right)^2} \label{odd_symmetry_eq1}  \\
    \end{equation}
    \begin{equation}
        -P_{RC}(-f) = - \frac{1}{T_b} \cdot \frac{\cos^2(-\pi f T_b)}{1 - \left({2\alpha (-f)}{T_b}\right)^2} \\
    \end{equation}  
    \begin{equation}
        -P_{RC}(-f) = - \frac{1}{T_b} \cdot \frac{\cos^2(-\pi f T_b)}{1 - \left({2\alpha f}{T_b}\right)^2} \label{odd_symmetry_eq2}
    \end{equation}
    Therefore, we can conclude by comparing equation1 \ref{odd_symmetry_eq1} to equation2 \ref{odd_symmetry_eq2} that $P_{RC}(f)$ exhibits odd symmetry with respect to $f = \frac{1}{T_b}$, as equation \ref{eq:odd_symmetry_condition} is satisfied.
\end{answer_box*}

\question
{
    We can recover ideal Nyquist pulse by plugging $\alpha  = 0$ to the RC pulse definition verify formally?
}
\begin{answer_box*}

    The raised cosine (RC) pulse is defined as:

    \begin{equation}
        P_{RC}(t) = \frac{1}{T_{b}} \cdot \frac{\cos\left(\pi\alpha t/T_{b}\right)}{\sin\left(\pi t/T_{b}\right)} \cdot \left(1-4\alpha^{2} \frac{t^{2}}{T^{2}}\right)
    \end{equation}

    To recover the ideal Nyquist pulse (Nyq), we need to set $\alpha = 0$ in this definition:

    \begin{equation}
        P_{Nyq}(t) = \frac{1}{T_{b}} \cdot \frac{\sin\left(\pi t/T_{b}\right)}{\pi t/T_{b}}
    \end{equation}

    Now, we need to verify whether the RC pulse with $\alpha = 0$ is equal to the ideal Nyquist pulse. Simplifying the expression, we have:

    \begin{equation}
        P_{RC}(t) \vert \alpha =0 = \frac{1}{T_{b}} \cdot \frac{1}{\sin\left(\pi t/T_{b}\right)}
    \end{equation}

    Comparing this with the definition of the ideal Nyquist pulse, we can see that it is indeed the same. Therefore, we can conclude that by setting $\alpha = 0$ in the definition of the raised cosine pulse, we can recover the ideal Nyquist pulse.
\end{answer_box*}

\question
{
    Prove the $P_RC$ in time domain?
}
\begin{answer_box*}
\end{answer_box*}

\question
{
    find $t_0$ such that $P_RC (t - t_0 ) < 0.001\ for\ t < 0$ ?
}
\begin{answer_box*}
    Since we are interested in finding $t_0$ such that $P_{RC}(t-t_0) < 0.001$ for $t<0$, we can restrict our attention to the range $t\in[-t_0,0]$. Therefore, we need to find $t_0$ such that:

\begin{equation}
    P_{RC}(t-t_0) < 0.001, \quad \text{for } -t_0 \leq t \leq 0
\end{equation}

Using the expression for $P_{RC}(t-t_0)$, we can rewrite the above inequality as:

\begin{equation}
    1 + \cos\left(\frac{\pi(t-t_0)}{T_b}\right) < 0.001, \quad \text{for } -t_0 \leq t \leq 0
\end{equation}

Since $\cos(\theta) \leq 1$ for all $\theta$, we can simplify the above inequality to:

\begin{equation}
    \frac{1}{T_b} < 0.001, \quad \text{for } -t_0 \leq t \leq 0
\end{equation}

which implies that

\begin{equation}
    t_0 > 1000T_b.
\end{equation}

Therefore, if we shift the raised cosine pulse to the right by $t_0 > 1000T_b$, then the pulse will satisfy $P_{RC}(t-t_0) < 0.001$ for $t<0$.
\end{answer_box*}

\question
{
    verify orthogonality of SRRC ?
}
\begin{answer_box*}
We aim to prove that the orthogonality of SRRC pulses is given by the fact that the integral of their product over the real line is zero for all non-zero $n$:


\begin{equation}
    \int_{-\infty}^{\infty} g_T(t)g_T(t-nT_b)dt = 0 \quad \text{for all } n \neq 0
\end{equation}


Substituting $g_T(t)$ and $g_T(t-nT_b)$ into the integral, we get:


\begin{equation}
\int_{-\infty}^{\infty} g_T(t)g_T(t-nT_b)dt = \int_{-\infty}^{\infty} \sin\left(\frac{\pi t}{T_b}\right)\cos\left(\frac{\pi \alpha t}{T_b}\right)\sin\left(\frac{\pi(t-nT_b)}{T_b}\right)\cos\left(\frac{\pi \alpha(t-nT_b)}{T_b}\right)dt
\end{equation}

\begin{align}
    &= \frac{1}{4}\int_{-\infty}^{\infty} \cos\left(\frac{\pi(t-u)}{T_b}\right)-\cos\left(\frac{\pi(t+u)}{T_b}\right)du \\
    &= \left[\left(\frac{\pi(t-u)}{T_b}\right)^2-\pi^2\alpha^2\right]\left[\left(\frac{2\alpha(t-u)}{T_b}\right)^2-1\right]+\frac{1}{4}\int_{-\infty}^{\infty} \cos\left(\frac{\pi(t-u)}{T_b}\right)+\cos\left(\frac{\pi(t+u)}{T_b}\right)du \\
    &= \left[\left(\frac{\pi(t-u)}{T_b}\right)^2-\pi^2\alpha^2\right]\left[\left(\frac{2\alpha(t+u)}{T_b}\right)^2-1\right]
\end{align}

Substituting $u=t-nT_b$, we get:

\begin{align}
    & \int_{-\infty}^{\infty} g_T(t)g_T(t-nT_b)dt = \frac{1}{4}\int_{-\infty}^{\infty} \cos\left(\frac{\pi u}{T_b}\right)-\cos\left(\frac{\pi(u+2nT_b)}{T_b}\right)du\\
    & = \left[\left(\frac{\pi u}{T_b}\right)^2-\pi^2\alpha^2\right]\left[\left(\frac{2\alpha u}{T_b}\right)^2-1\right]+\frac{1}{4}\int_{-\infty}^{\infty} \cos\left(\frac{\pi u}{T_b}\right)+\cos\left(\frac{\pi(u+2nT_b)}{T_b}\right)du\\
    & = \left[\left(\frac{\pi u}{T_b}\right)^2-\pi^2\alpha^2\right]\left[\left(\frac{2\alpha(u+2nT_b)}{T_b}\right)^2-1\right] 
\end{align}

The first integral is zero because the integrand is an odd function of $u$. The second integral is also zero because the integrand is an even function of $u$ and the limits of integration are symmetric around $u=0$.

Therefore, we have shown that the integral of the product of two SRRC pulses, $g_T(t)$ and $g_T(t-nT_b)$, over the entire real line is zero for all $n\neq 0$. This proves that SRRC pulses are orthogonal to each other.
\end{answer_box*}

\question
{
    If $g_T (t)$ is delayed by multiples of $T_b$ , then the area of multiplication = 0. Why is this important for detection?
}
\begin{answer_box*}
\begin{itemize}
    \item The property that the area of multiplication is zero when the filter impulse response $g_T (t)$ is delayed by multiples of the symbol period $T_b$ is crucial for detection because it prevents inter-symbol interference (ISI).

    \item ISI occurs when symbols from adjacent time intervals overlap in time, causing distortion in the received signal. By ensuring that the area of multiplication is zero for delays that are multiples of T b , we effectively eliminate ISI.
    
    \item This is achieved by shifting the filter to a different time interval where it has no overlap with adjacent symbols. As a result, each symbol is sampled at a time when it has no interference from neighboring symbols, making it easier to detect correctly.
\end{itemize}
\end{answer_box*}

\question
{
    What is the maximum bit rate for any pulse?
}
\begin{answer_box*}
    $R_b = 2W$
\end{answer_box*}

\question
{
    Plot T.D \& F.D pulse using MATLAB
}
\begin{answer_box*}
\end{answer_box*}
\question
{
    How do we know that there wont be any uncontrolled ISI in addition to ISI from $a_{k-1}$ ?
}
\begin{answer_box*}
    \begin{itemize}
        \item To prevent uncontrolled inter-symbol interference (ISI) beyond that caused by the previous symbol (k-1), the pulse shape and system bandwidth must be carefully designed.

        \item The pulse shape should be selected to eliminate ISI at the decision points and minimize energy outside the decision window. This ensures that ISI from adjacent symbols is negligible.
        
        \item Therefore, the system bandwidth should be limited to prevent frequency-dependent attenuation of the transmission medium from causing ISI. This ensures that symbols are transmitted with minimal distortion and that ISI from distant symbols is also minimized.
    \end{itemize}
\end{answer_box*}
\question
{
    How $\hat{a}_k = x_k - \hat{a}_{k-1}$ is equivalent to :    
\begin{equation}
    \hat{a}_k =
    \begin{cases}
        -1             &\text{if } x_k = -2 \\
        1              &\text{if } x_k =  2 \\
        -\hat{a}_{k-1} &\text{if } x_k =  0
    \end{cases}
\end{equation}

    
}
\begin{answer_box*}
\begin{itemize}
    \item  When $x_k = -2$: Substituting $x_k = -2$ into the equation gives $\hat{a}_k = -2 - \hat{a}_{k-1}$, from the definition of $\hat{a}_k$ in the original equation, we have $\hat{a}_k = -1$, so substituting this value gives $\hat{a}_{k-1} = -2 - (-1) = -1$. Therefore, we have $\hat{a}_k = -1$ for $x_k = -2$.
    \item  When $x_k = 2$: Substituting $x_k = 2$ into the equation gives $\hat{a}_k = 2 - \hat{a}_{k-1}$, from the definition of $\hat{a}_k$ in the original equation, we have $\hat{a}_k = 1$, so substituting this value gives $\hat{a}_{k-1} = 2 - 1 = 1$. Therefore, we have $\hat{a}_k = 1$ for $x_k = 2$.
    \item When $x_k = 0$: Substituting $x_k = 0$ into the equation gives $\hat{a}_k = 0 - \hat{a}_{k-1}$, which can be rewritten as $\hat{a}_k = -\hat{a}_{k-1}$. Therefore, we have $\hat{a}_k = -\hat{a}_{k-1}$ for $x_k = 0$.  
\end{itemize}
\end{answer_box*}
\question
{
    Would detection differ if $w_0 = 1$?
}
\begin{answer_box*}
    No, since the duobinary sequence will be inverted but detection is the same of binary input data
\end{answer_box*}

\question
{
    Another version of correlative encoding is called modified duobinary, where
    \begin{equation}
        \begin{cases}
        P (n {T}_b) &= -1 \quad \text{if } n = 0 \quad \text{Repeat the analysis \& find P (t), P (f ). Explain why it is preferrable} \\
        P (n {T}_b) &= 1 \quad \text{if } n = 0 \\
        P (n {T}_b) &= 0 \quad \text{otherwise} 
        \end{cases}
    \end{equation}
    in channels that doesnt pass DC levels.
}
\begin{answer_box*}

    Modified duobinary encoding is a type of correlative encoding that modifies the duobinary encoding scheme. 
    Specifically, it changes the polarity of the -1 symbol.\\
    The autocorrelation function of the modified duobinary signal is given by:
\begin{equation}
    R(\tau ) = \int_{-\infty}^{\infty} P (t)P (t - \tau )dt
\end{equation}

where,

\begin{equation}
    R(\tau ) = \int_{-\infty}^{\infty} P (t)P (t)dt + \int_{-\infty}^{\infty} P (t)P (t - \tau )dt \\
     = T_b + (-1)^{\frac{\tau}{T_b}}T_b \\
\end{equation}

\begin{equation}
    \begin{cases}
        \frac{1}{2} & \text{if } t = 0 \\
        \frac{1}{4} & \text{if } t = \pm T_b \\
        0 & \text{otherwise.}
    \end{cases}
\end{equation}

\begin{equation}
    F (f ) = \left(\frac{1}{2} + \frac{3}{4} e^{-j2\pi f T b } + \frac{1}{4} e^{j2\pi f T b}\right).
\end{equation}

\begin{equation}
    S(f) = |F(f)|^2 = \frac{1}{2} \left(1 + \cos(2\pi fT_b)\right)
\end{equation}

By comparing this with the PSD of the standard duobinary signal, we can see that the modified duobinary signal has a wider bandwidth. This is due to the presence of a DC component in the modified duobinary signal.
Therefore, the modified duobinary signal is preferable in channels that do not pass DC levels. This is because the wider bandwidth of the modified duobinary signal will allow it to pass through channels that do not pass DC levels more easily.

\end{answer_box*}
% ========================== end Question of Note 2===============================
% --------------------------------------------------------------------------------
% ========================== start Question of Note 3===============================
% --------------------------------------------------------------------------------
\section{Note 3}
\question
{
    If $H_c(f)$ is a non-distorting channel, do we need an equalizer?
}
\begin{answer_box*}
    A non-distorting channel with a flat frequency response does not require an equalizer as it transmits the signal without distortion or attenuation. However, if the non-distorting channel has a frequency-dependent gain, an equalizer may be necessary to correct for the gain variation and ensure the signal's fidelity.
\end{answer_box*}

\question
{
    What is a Decision Feedback Equalizer (DFE)?
}
\begin{answer_box*}
In a dispersive channel, a Decision Feedback Equalizer (DFE) uses feedback from the receiver's decision circuit to adaptively adjust its taps and remove Inter-Symbol Interference (ISI). ISI occurs when a symbol interferes with adjacent symbols due to the channel's varying gain and phase response with frequency.

The DFE consists of a feedforward filter for traditional equalization and a feedback filter to remove residual ISI. The feedback signal is obtained by delaying the received signal and multiplying it by the receiver's decision. This signal is then subtracted from the input to the equalizer.

The DFE adapts the feedback filter's tap coefficients using an adaptive algorithm, such as LMS or RLS, to minimize the error between the received signal and the decision feedback signal. This adaptive adjustment effectively compensates for ISI.

DFE provides excellent performance in dispersive channels with significant ISI. However, it has a more complex implementation and higher computational complexity compared to other equalization techniques.
\end{answer_box*}

\question
{
    Why is this equation $y_n = \sum_(k=-L_2)^(P) k a_n-k + n^e$ exactly the same as $y_n = \sum_(k=-L_2)^(L_1) a_k P_(n-k) + n^e$?
}
\begin{answer_box*}
    Both equations represent the convolution of an FIR filter with an input signal, where:

\begin{itemize}
        \item $y_n$ is the output of the filter at time n
        \item $x_n$ is the input signal at time n
        \item $a_k$ is the filter coefficient at tap k
        \item $L_1$ and $L_2$ are the lower and upper limits of the filter taps, respectively
        \item $e_n$ is any additive noise present in the signal
\end{itemize}

The two equations are equivalent and can be used interchangeably. The choice of which equation to use depends on the specific application or context in which the filter is being used. For example, if the input signal is known in advance and the filter coefficients are being determined, Equation 2 may be more useful. Conversely, if the filter coefficients are known and the input signal is being processed, Equation 1 may be more convenient.
\end{answer_box*}

\question
{
    What is pseudonoise (PN) sequence? describe its generation \& autocorrelation function ?
}
\begin{answer_box*}
    \begin{itemize}
        \item \textbf {Pseudonoise (PN) Sequence}:
    
        A pseudonoise (PN) sequence is a deterministic sequence of binary values that appears random. PN sequences are used in various applications, including digital communications, cryptography, and radar.
    
        \item \textbf {Generation}:
    
        PN sequences are generated using linear feedback shift registers (LFSRs). An LFSR consists of a shift register with feedback taps connected to an XOR gate. The feedback taps are chosen based on the desired length and properties of the PN sequence.
    
        \item \textbf {Autocorrelation Function}:
    
        The autocorrelation function of a PN sequence is given by:
    
        \begin{equation}
        R(\tau) \equiv 
        \begin{cases}
        N,     & \text{if } \tau=0 \\
        0,     & \text{if } \tau \neq 0
        \end{cases}
        \end{equation}
        where:

        \begin{itemize}
            \item  {N is the length of the PN sequence}
            \item  {$\tau$ is the time delay between two subsequences of the PN sequence}
        \end{itemize}
        The autocorrelation function shows that a PN sequence is uncorrelated with itself at any time delay except for zero. This property makes PN sequences useful for applications where random-like signals are required.
    
    \end{itemize}\end{answer_box*}

\question
{
    What is the importance of using a sequence with ACF similar to noise ACF (i.e,delta function)?
}
\begin{answer_box*}
Using a pseudonoise (PN) sequence with an ideal autocorrelation function (i.e., a maximum-length sequence) in a communication system can improve the system's performance in terms of noise immunity and spectral efficiency.

\textbf{Noise Immunity}
When a PN sequence is used as a spreading code in a direct-sequence spread spectrum (DSSS) system, the signal is spread out over a wide frequency band. This makes it more difficult for an eavesdropper or a jammer to selectively interfere with the signal, because any interference will be spread out over the entire band. Additionally, the PN sequence can be regenerated at the receiver using the same feedback polynomial as the transmitter, which provides a means of despreading the signal and recovering the original data in the presence of noise.

\textbf{Spectral Efficiency}
The use of a PN sequence with an ideal autocorrelation function allows the signal to spread out over a wider bandwidth, which increases the system's spectral efficiency. This is because the sequence is uncorrelated with itself at any time delay except for a delay of zero. This property allows the signal to occupy a wider bandwidth without causing intersymbol interference.
\end{answer_box*}

\question
{
    Complete the steps of integral ${\sigma_{n}}^2 = N_0 (1 + x^2 ){\cos^2}(\pi x/2)dx$?
}
\begin{answer_box*}
    To complete the steps of the integral ${\sigma_{n}}^2 = N_0(1 + x^2){\cos^2}(\pi x/2)dx$, follow these steps:

    \begin{equation}\sigma^2 = N_0\left(\frac{5}{6} - \frac{2}{\pi^2}\right)\end{equation}

    where:

    \begin{align}
    \int_0^1 \frac{1}{2}(1 + x^2)dx &= \frac{5}{6} \tag{1}\\
    \int_0^1 \frac{1}{2}(1 + x^2)\cos(\pi x)dx &= -\frac{2}{\pi^2} \tag{2}
    \end{align}

    \begin{itemize}
        \item (1) Integrate by parts:
        \begin{align}\int_0^1 \frac{1}{2}(1 + x^2)dx &= \left[\frac{x}{2} + \frac{x^3}{6}\right]_0^1 = \frac{5}{6}\end{align}
        \item (2) Integrate by parts:
        \begin{align}\int_0^1 \frac{1}{2}(1 + x^2)\cos(\pi x)dx &= \left[-\frac{1}{2}(1 + x^2)\sin(\pi x)\right]_0^1 - \frac{1}{\pi}\int_0^1 \pi x \sin(\pi x)dx \end{align}
        \begin{align}-\int_0^1 \pi x \sin(\pi x)dx &= \left[-\frac{x\sin(\pi x)}{\pi}\right]_0^1 + \frac{1}{\pi^2}\int_0^1 \cos(\pi x)dx \end{align}
        \begin{align}-\int_0^1 \cos(\pi x)dx &= \left[\frac{1 - \cos(\pi x)}{(\pi)^2}\right]_0^1 = \frac{1}{\pi^2}\end{align}
    \end{itemize}

    Therefore, the final result is:

    \begin{equation}\sigma^2 = N_0\left(\frac{5}{6} - \frac{2}{\pi^2}\right)\end{equation}

\end{answer_box*}

\question
{
    What is the advantage of having $\tau < T_b$?
}
\begin{answer_box*}
    Having $\tau < T_b$ in a ZF equalizer has the following advantages:

\begin{itemize}
    \item {\textbf{Better performance in frequency-selective channels:}} By having a tap spacing less than $T_b$, the ZF equalizer can better capture the multi-path components of the channel and reduce ISI to a greater extent, resulting in better performance in frequency-selective channels.
    \item {\textbf{Improved noise rejection:}} The ZF equalizer can better distinguish between noise and signal components that arrive at different delays when the tap spacing is less than $T_b$, allowing it to better reject noise and improve the signal-to-noise ratio (SNR) of the received signal.
    \item {\textbf{Lower computational complexity:}} The smaller tap spacing allows each tap to have a shorter impulse response, which leads to lower computational complexity in the equalizer.
\end{itemize}
\end{answer_box*}

\question
{
    Why we dont write for $n \neq 0$ ?
}
\begin{answer_box*}
    As we have N taps only, so the total is (2N + 1).
\end{answer_box*}

\question
{
    \[
    q(nT_b) = \sum_{m=-2}^2 c_m P(nT_b - mT_{2b}) =
     \begin{cases}
    1 & \text{if } n = 0 \\
    0 & \text{if } n = \pm 1 , \pm 2 
    \end{cases}
    \] write explicit equations?
}
\begin{answer_box*}
    For $n = 0$, we have:
    \begin{equation}
        q(0) = c_{-2}P(-\frac{T_{b}}{2}) + c_{-1}P(-\frac{T_{b}}{2}) + c_0P(0) + c_1P(-\frac{T_{b}}{2}) + c_2P(-T_{b}) = 1
    \end{equation}

    For $n = 1$, we have:
    \begin{equation}
        q(T_b) = c_{-2}P(\frac{T_{b}}{2}) + c_{-1}P(T_{b}) + c_0P(-\frac{T_{b}}{2}) + c_1P(-\frac{T_{b}}{2}) + c_2P(-T_{b}) = 0
    \end{equation}

    For $n = -1$, we have:
    \begin{equation}
        q(T_b) = c_{-2}P(\frac{T_{b}}{2}) + c_{-1}P(- T_{b}) + c_0P(\frac{T_{b}}{2}) + c_1P(\frac{T_{b}}{2}) + c_2P(-T_{b}) = 0
    \end{equation}

    For $n = 2$, we have:
    \begin{equation}
        q(T_b) = c_{-2}P(\frac{T_{b}}{2}) + c_{-1}P(2T_{b}) + c_0P(-\frac{T_{b}}{2}) + c_1P(-\frac{T_{b}}{2}) + c_2P(-2T_{b}) = 0
    \end{equation}

    For $n = -2$, we have:
    \begin{equation}
        q(T_b) = c_{-2}P(\frac{T_{b}}{2}) + c_{-1}P(-2T_{b}) + c_0P(\frac{T_{b}}{2}) + c_1P(\frac{T_{b}}{2}) + c_2P(-2T_{b}) = 0
    \end{equation}
\end{answer_box*}

\question
{
    What is Toeplitz matrix ? 
}
\begin{answer_box*}
A Toeplitz matrix is a square matrix in which the elements along each diagonal from the main diagonal to the secondary diagonals are constant. In other words, the elements of a Toeplitz matrix are constant along its antidiagonals.


\begin{equation}
\begin{bmatrix}
a_0 & a_1 & a_2 & \cdots & a_{n-1} \\
a_{-1} & a_0 & a_1 & \cdots & a_{n-2} \\
a_{-2} & a_{-1} & a_0 & \cdots & a_{n-3} \\
\vdots & \vdots & \vdots & \ddots & \vdots \\
a_{-(n-1)} & a_{-(n-2)} & a_{-(n-3)} & \cdots & a_0
\end{bmatrix}
\end{equation}
where $a_i$ is the constant value of the $i$th diagonal.

\end{answer_box*}

\question
{
    What about $q(3T_b )$?
}
\begin{answer_box*}

$$q(3T_b) = c_{-1}P(4T_b) + c_0P(3T_b) + c_1P(2T_b)$$

Since no response is given for P(4T b ) \& P(3T b ), we can consider them zero:

$$q(3T_b) = c_1P(2T_b) = 0.1 * 0.3448 = 0.03448$$
\end{answer_box*}

\question
{

}
\begin{answer_box*}
    Expanded expression:

    $$\sum_m\sum_k{c_m,c_k \in \{-1, 0, 1\}} c_m c_k Ry(k - m) =$$

    $$c_{-1} c_{-1} R_y(-1 - (-1)) + c_{-1} c_0 R_y(0 - (-1)) + c_{-1} c_1 R_y(1 - (-1)) +$$

    $$c_0 c_{-1} R_y(-1 - 0) + c_0 c_0 R_y(0 - 0) + c_0 c_1 R_y(1 - 0) +$$

    $$c_1 c_{-1} R_y(-1 - 1) + c_1 c_0 R_y(0 - 1) + c_1 c_1 R_y(1 - 1)$$

    Simplified expression:

    $$= c_{-1} c_{-1} R_y(0) + c_{-1} c_0 R_y(1) + c_{-1} c_1 R_y(2) +$$

    $$c_0 c_{-1} R_y(-1) + c_0 c_0 R_y(0) + c_0 c_1 R_y(1) +c_1 c_{-1} R_y(-2) + c_1 c_0 R_y(-1) + c_1 c_1 R_y(0)$$
\end{answer_box*}

\question
{
    What happens if $\mu$ is large?
}
\begin{answer_box*}
    at $\mu {\rightarrow}\infty \implies$ :
\begin{itemize}
    \item \textbf{Overshooting the Minimum}:  Algorithm overshoots minimum, leading to divergence.
    
    \item \textbf{Divergence}: Loss function increases or oscillates without convergence.
        
    \item \textbf{Slow Convergence or Oscillation}: Algorithm converges slowly or oscillates around minimum.
        
    \item \textbf{Failure to Converge}:  Algorithm fails to converge altogether.
        
\end{itemize}
    \textbf{Mitigation Strategies:}\\
    - Tune learning rate appropriately.\\
    - Use learning rate decay or adaptive learning rates (e.g., Adagad optimizer).\\
    - Employ line search methods to dynamically adjust step size.
\end{answer_box*}
\question
{
    Block diagram of in p.g67 corresponds to training phase or decision- directed phase?
}
\begin{answer_box*}
\begin{itemize}
    \item \textbf {Training Phase:}
    
    1- The equalizer uses known training symbols to estimate the channel response.\\
    2- The estimated channel response is used to update the equalizer coefficients.\\
    3- The training phase is typically used initially to obtain an accurate estimate of the channel.
    
    \item \textbf {Decision-Directed Phase:}
    
    1- The equalizer uses decisions made on received symbols to update its coefficients.\\
    2- The decisions are based on the current estimate of the channel response.\\
    3- The decision-directed phase is used to track and adapt to changes in the channel.   
\end{itemize}


The key difference between the two phases is the source of the information used to update the equalizer coefficients: known training symbols in the training phase and decisions on received symbols in the decision-directed phase.\\
In practice, the equalizer may start in the training phase to obtain an initial estimate of the channel response. Once the channel has been estimated, the equalizer switches to the decision-directed phase to track and adapt to changes in the channel while making decisions on the received symbols.
\end{answer_box*}
\question
{
    Search for : Maximum likelihood sequence equalizer?
}
\begin{answer_box*}
    The Maximum Likelihood Sequence Equalizer (MLSE) is a technique used for equalization in communication systems. It aims to mitigate the effects of intersymbol interference (ISI) caused by a dispersive channel. MLSE takes into account the entire received sequence of symbols to make optimal decisions about the transmitted symbols.\\
    The MLSE equalizer operates based on the principle of maximum likelihood estimation, which means it seeks to find the most likely sequence of transmitted symbols given the received sequence. The equalizer considers all possible transmitted symbol sequences and computes their probabilities based on the received sequence and the estimated channel response.\\
    Here is a high-level overview of the MLSE process:

\begin{itemize}
    \item \textbf{Channel Estimation}: MLSE requires an estimation of the channel impulse response. This can be obtained through techniques like pilot symbols, training sequences, or blind estimation algorithms.
    \item \textbf{State Metric Calculation}: The MLSE equalizer maintains a set of internal states that represent the possible past symbol sequences. The equalizer calculates a metric for each state based on the received signal and the estimated channel response. The metric represents the likelihood of each state given the received signal.
    \item \textbf{Path Metric Update}: The MLSE equalizer updates the metrics associated with each state using the received signal and the estimated channel response. This update takes into account the likelihood of transitioning between states based on the channel response.
    \item \textbf{Path Pruning}: To reduce computational complexity, the MLSE equalizer prunes less likely paths by discarding states with lower metrics, keeping only the most likely states. This pruning helps to focus on the more probable symbol sequences.
    \item \textbf{Decision Making}: After processing the entire received sequence, the MLSE equalizer makes a decision on the transmitted symbols based on the most likely state.    
\end{itemize}
    MLSE can be computationally intensive, as it considers all possible symbol sequences. However, it provides near-optimal equalization performance in the presence of severe ISI. Various optimization techniques, such as the Viterbi algorithm, can be employed to efficiently implement the MLSE equalizer.\\
    Its worth noting that MLSE is often used in scenarios where the channel characteristics are time-varying or unknown and when the transmitted symbols are not constrained to a specific set.
\end{answer_box*}
\question
{
    Search for : Decision Feedback equalizers?
}
\begin{answer_box*}
    DFE is an equalization technique that mitigates intersymbol interference (ISI) by utilizing both past and future symbols. It consists of two main components: the feedforward filter and the feedback filter.\\
    The feedforward filter processes the received signal to minimize distortion caused by the direct channel path. The feedback filter uses decisions made on past symbols to estimate and cancel interference from future symbols.\\
    DFE operates in a feedback loop:

\begin{itemize}
    \item \textbf{Feedforward Filtering}: The received signal is processed by the feedforward filter.
    \item \textbf{Decision Making}: A decision is made on the transmitted symbol based on the feedforward filter output.
    \item \textbf{Feedback Filtering}: The decision on the previous symbol is processed by the feedback filter to cancel interference from future symbols.
    \item \textbf{Decision Feedback}: The feedback filter output is combined with the previous symbol decision to refine the current symbol decision.
    \item \textbf{Error Calculation}: The refined decision is compared to the actual received symbol to update the feedforward and feedback filter coefficients.
\end{itemize}
    DFE iteratively refines symbol decisions by incorporating information from both past and future symbols. It is effective in scenarios with severe ISI and time-varying or unknown channel characteristics. However, it may introduce noise enhancement, which can be mitigated through proper design and adaptation techniques.
\end{answer_box*}
% ========================== end Question of Note 3===============================
% --------------------------------------------------------------------------------

% ==============================================================================
%
% Chapter 2
%
% ==============================================================================

\chapter{Introduction to information theory}
\noindent\makebox[\linewidth]{\rule{\textwidth}{0.4pt}}

\question
{
}
\begin{answer_box*}
\end{answer_box*}
\question
{
}
\begin{answer_box*}
\end{answer_box*}
\question
{
}
\begin{answer_box*}
\end{answer_box*}
\question
{
}
\begin{answer_box*}
\end{answer_box*}
\question
{
}
\begin{answer_box*}
\end{answer_box*}

% ==============================================================================
%
% Reference
%
% ==============================================================================

\chapter{References}
\noindent\makebox[\linewidth]{\rule{\textwidth}{0.4pt}}

\label{finalpage}
\end{document}