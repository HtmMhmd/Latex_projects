\documentclass[14pt,a4paper]{report}
\usepackage{tocloft}
\usepackage{hyperref}
\usepackage{lastpage}
\usepackage{xcolor}
\usepackage{amsmath}
\usepackage{fancyhdr}
\usepackage{enumitem}
\usepackage{tabularx}
\usepackage[skip=2pt,font=small]{caption}
\usepackage{environ}
\usepackage{mdframed}
\usepackage{graphicx}
\usepackage{ntheorem}


% Booleans to show answers and to ask students to return the question form
\newif\ifshowanswers

% ==============================================================================
% Question command
%
\newcounter{question}
\makeatletter
\@addtoreset{question}{section} % reset question counter at every section
\makeatother
\newcommand*\question{%
\stepcounter{question}%
\paragraph{Question \thesection.\thequestion}}

% ==============================================================================
% AnsweR_boxes
%
\mdtheorem[outerlinewidth=2,roundcorner=10pt,
leftmargin=0,rightmargin=0,
backgroundcolor=yellow!40,outerlinecolor=blue!70!black,
innertopmargin=\topskip,splittopskip=\topskip,
ntheorem=true,]{answer_box}{Answer}[section]

\NewEnviron{answer}{
\ifshowanswers
\begin{answer_box*}
\BODY
\end{answer_box*}
\fi}

% ==============================================================================
% Show answers?
%
\showanswerstrue
% \showanswersfalse


% ==============================================================================
%
% Course information
%
% ==============================================================================

\newcommand{\coursename}{Introduction to Digital Communication}
\newcommand{\coursecode}{ECE 382}

\newcommand{\assigntype}{Portfolio}
\newcommand{\teacher}{Hatem Maohamed Ahmed (\texttt{20010447})}



% ==============================================================================
%
% Margins, header and footer
%
% ==============================================================================
\setlength{\topmargin}{0cm}
\setlength{\textheight}{9.25in}
\setlength{\oddsidemargin}{0.0in}
\setlength{\evensidemargin}{0.0in}
\setlength{\textwidth}{16cm}
\pagestyle{fancy}
\cfoot{\footnotesize{Page \thepage \ of \pageref{finalpage}
-- \chaptername \ \thechapter \ \sectionautorefname \ \thesection}}
\renewcommand{\headrulewidth}{0pt}
\renewcommand{\footrulewidth}{0pt}

\begin{document}

% ==============================================================================
%
% Header
%
% ==============================================================================

\noindent\makebox[\linewidth]{\rule{\textwidth}{0.4pt}}
\begin{center}
\Large \textbf{\coursename} (\coursecode)
\end{center}

\begin{center}
\large \assigntype{} \\
\vspace{3mm}
\end{center}

\begin{center}
\teacher\\

\end{center}

\noindent\makebox[\linewidth]{\rule{\textwidth}{0.4pt}}



% Portfolio Content
\tableofcontents

% ==============================================================================
%
% Chapter 1
%
% ==============================================================================


\chapter{Transmission through band limited channels}
\noindent\makebox[\linewidth]{\rule{\textwidth}{0.4pt}}

% --------------------------------------------------------------------------------

% ==========================begin Question of Note 1==========================
\section{Note 1}
\question 

{
    What are the bandwidths for different communication channels?
}

\begin{answer_box*}
{
\begin{tabular}{|l|l|}
    \hline
    Communication Channel & Bandwidth \\
    \hline
    \textbullet\ Telephone channels & \textbullet\ 4 kHz \\
    \textbullet\ Microwave LOS radio channel & \textbullet\ 40 MHz (2.4 GHz) to 2.16 GHz (60 GHz) \\
    \textbullet\ Satellite channel & \textbullet\ 36 MHz (C-band) to 500 MHz (Ka-band) \\
    \textbullet\ Underwater acoustic channel & \textbullet\ A few tens of Hz to a few kHz \\
    \hline
\end{tabular}
}

\paragraph{Telephone channels:}

The bandwidth of a telephone channel is typically 4 kHz. This bandwidth is sufficient to transmit voice signals with good quality, but not enough for transmitting high-quality audio or video.

\paragraph{Microwave LOS radio channel:}

The bandwidth of microwave LOS radio channels can vary depending on the frequency band used. For example, the bandwidth of a 2.4 GHz microwave link can be up to 40 MHz, while a 60 GHz microwave link can have a bandwidth of up to 2.16 GHz. Microwave LOS radio channels are often used for high-speed data transmission, such as wireless broadband internet.

\paragraph{Satellite channel:}

The bandwidth of a satellite channel also varies depending on the type of satellite and the frequency band used. For example, a typical satellite channel in the C-band has a bandwidth of around 36 MHz, while a Ka-band satellite channel can have a bandwidth of up to 500 MHz. Satellite channels are often used for long-distance communication, such as satellite TV and satellite internet.

\paragraph{Underwater acoustic channel:}

The bandwidth of an underwater acoustic channel is much lower than that of other communication channels due to the high attenuation of sound waves in water. The bandwidth of an underwater acoustic channel can range from a few tens of Hz to a few kHz, depending on the distance between the transmitter and the receiver, and the frequency used. Underwater acoustic channels are often used for underwater communication, such as underwater sonar and underwater modems.

\end{answer_box*}

\question
{
    Why is a wireless channel time-varying?
}
\begin{answer_box*}

    Wireless channels are time-varying because the electromagnetic waves that carry the wireless signals can experience a variety of changes as they propagate through the environment. These changes can occur due to several factor expressed below: \\

    \begin{tabular}{|l|p{0.7\linewidth}|p{0.3\linewidth}|}
            \hline
            Reflection & When a wireless signal encounters an obstacle, it can reflect off the surface and reach the receiver via a longer path, leading to time-varying signal interference. \\
            \hline
            Diffraction & When a wireless signal encounters a large obstacle, it can bend around it and reach the receiver, causing signal fading due to constructive and destructive interference. \\
            \hline
            Scattering & When a wireless signal encounters small objects, it can scatter in multiple directions, resulting in time-varying signals and multipath interference. \\
            \hline
            Doppler effect & When the transmitter or receiver is in motion, the frequency of the received signal can shift due to the Doppler effect, causing time-varying signal strength. \\
            \hline
    \end{tabular}

\end{answer_box*}

\question
{
    Using MATLAB, generate signal $x(t) = sinc(t/0.01)$ , $H c (f ) = e ^{( - j * 2 * \pi * (f + 10f 2 ))}$ plot output if $x(t)$ passes through $H c (f )$ ?
}
\begin{answer_box*}
    \begin{verbatim}
    % Define the sampling frequency
    fs = 1000;

    % Define the time vector
    t = -1:1/fs:1;

    % Define the signal x(t)
    x = sinc(t/0.01);

    % Define the frequency vector
    f = -fs/2:fs/length(x):fs/2-fs/length(x);

    % Define the channel frequency response Hc(f)
    Hc = exp(-1j * 2 * pi * (f + 10 * f.^2));

    % Compute the output of the channel
    y = x .* Hc;

    % Plot the input and output signals
    figure;
    subplot(2,1,1);
    plot(t, x);
    title('Input Signal x(t)');
    xlabel('Time (t)');
    ylabel('Amplitude');

    subplot(2,1,2);
    plot(f, abs(fftshift(fft(y))));
    title('Output Signal Y(f)');
    xlabel('Frequency (f)');
    ylabel('Magnitude');
    \end{verbatim}
    
\end{answer_box*}

\question
{
    Do we need to have zero ISI at each time instant? or just at sampling time?
}
\begin{answer_box*}
    No, it is only necessary to have zero ISI at the sampling time.
\end{answer_box*}

\question{
    Why \(g_T(t) \ast h_c(t)\)?
}
\begin{answer_box*}
    Because the received signal is distorted by the channel response \(h_c(t)\) as it propagates through the communication channel. This distortion can cause the transmitted pulses to spread in time and overlap with each other, leading to inter-symbol interference (ISI). By designing the receiver filter to be matched to \(g_T(t) \ast h_c(t)\), we can effectively remove the ISI and improve the detection of the transmitted symbols.
\end{answer_box*}

\question{
    If \(a_k \in \{1, -1\}\), what should be \(V_{th}\)?
}
\begin{answer_box*}
    In a binary communication system with binary antipodal signaling, the threshold voltage \(V_{th}\) should be set to 0. This ensures that the decision boundary is midway between the two possible transmitted signal values (+1 and -1).
\end{answer_box*}

\question{
Why do we need to have ISI = 0 only at the sampling time?
}
\begin{answer_box*}
    ISI must be zero at the sampling time to ensure accurate symbol detection. Equalization techniques adjust the received signal to minimize ISI at this critical moment.
\end{answer_box*}

\question
{
    Why is optimal sampling at the maximum eye opening?
}
\begin{answer_box*}
    Optimal sampling occurs at the maximum eye opening because:\\
    • Symbols are most separated, reducing ISI effects. \\ 
    • Signal-to-noise ratio (SNR) is highest, improving receiver performance and reducing symbol errors.
\end{answer_box*}

\question
{
    Sketch the eye pattern in case the bandwidth of the channel is $\infty$.
}
\begin{answer_box*}

    With infinite bandwidth, ISI is eliminated, resulting in a wide-open eye pattern with distinct horizontal lines and vertical transitions representing symbols.

\end{answer_box*}

\question
{
    Why outside sampling times, overlapping pulses is of no practical significance?
}
\begin{answer_box*}
    Outside the sampling times, overlapping pulses do not contribute to the received signal because the sampler only takes samples at specific time intervals. The only concern is the pulse shape at the sampling times, which determines the distortion due to ISI.
\end{answer_box*}

\question 
{
    Why filters need to be causal?
}
\begin{answer_box*}
    Causality ensures that the filter has a physical realizable implementation. A causal filter is one whose output depends only on present and past input values. This means that the filter can be implemented in real-time, without requiring a delay line or other mechanism to store future values of the input signal.
\end{answer_box*}

\question 
{
    Why it is required to have aliasing between replicas of $P(f)$?
}
\begin{answer_box*}
    Aliasing between replicas of $P(f)$ is required to satisfy the Nyquist criterion and have constant amplitude = Tb by overlapping.\\
    The Nyquist criterion states that the sampling rate must be at least twice the highest frequency component of the signal to avoid aliasing.\\
    By overlapping the replicas of $P(f)$, we can ensure that the sampled signal has a constant amplitude. This is important for maintaining signal integrity and minimizing distortion.
\end{answer_box*}

\question 
{
    Verify that $P (nT_b ) = \delta (n)$ for $sinc(t/T_b )$?
}
\begin{answer_box*}
Proof:

We begin by taking the Fourier transform of the sinc function:
\begin{equation}
\left(\mathcal{F}[\text{sinc}(t/T_b)]\right) = \int_{-\infty}^{\infty} \frac{\sin(t/T_b)}{T_b} e^{-j2\pi ft} dt
\end{equation}
We then use the Fourier series expansion of the exponential function:
\begin{equation}
e^{-j2\pi ft} = \sum_{k=-\infty}^{\infty} \frac{(-1)^k}{j2\pi k} e^{-j2\pi kt}
\end{equation}
Substituting the Fourier series expansion into the integral, we get:
\begin{equation}
\mathcal{F}[\text{sinc}(t/T_b)] = \sum_{k=-\infty}^{\infty} \frac{(-1)^k}{j2\pi k} \int_{-\infty}^{\infty} \frac{\sin(t/T_b)}{T_b} e^{-j2\pi (k+f)t} dt
\end{equation}
Interchanging the order of integration and summation, we have:
\begin{equation}
\mathcal{F}[\text{sinc}(t/T_b)] = \sum_{k=-\infty}^{\infty} \frac{(-1)^k}{j2\pi k} \int_{-\infty}^{\infty} \frac{\sin(t/T_b)}{T_b} e^{-j2\pi (k+f)t} dt
\end{equation}
Evaluating the integral using the sifting property of the sinc function, we have:
\begin{equation}
\mathcal{F}[\text{sinc}(t/T_b)] = \sum_{k=-\infty}^{\infty} \frac{(-1)^k}{j2\pi k} \delta(k+f/T_b)
\end{equation}
Substituting $f = -k/T_b$ into the sum, we obtain:
\begin{equation}
\mathcal{F}[\text{sinc}(t/T_b)] = \sum_{k=-\infty}^{\infty} \frac{(-1)^k}{j2\pi k} \delta(k-f/T_b)
\end{equation}
Taking the inverse Fourier transform of both sides, we have:
\begin{equation}
\text{sinc}(t/T_b) = \sum_{k=-\infty}^{\infty} \frac{(-1)^k}{j2\pi k} \delta(k-f/T_b)
\end{equation}
Evaluating the inverse Fourier transform at $t = nT_b$, we obtain:
\begin{equation}
\text{sinc}(nT_b) = \sum_{k=-\infty}^{\infty} \frac{(-1)^k}{j2\pi k} \delta(k-n)
\end{equation}
Simplifying the sum, we have:
\begin{equation}
\text{sinc}(nT_b) = \frac{(-1)^n}{j2\pi n}
\end{equation}
Comparing the result to the definition of the Dirac delta function, we can see that:
\begin{equation}
P(nT_b) = \delta(n) \quad \text{for} \quad \text{sinc}(t/T_b)
\end{equation}
\end{answer_box*}

\question 
{
    Calculate min BW needed for telephone system using PCM with sampling rate of 8 kHz \& 256 quantization level.
}
\begin{answer_box*}
    To calculate the minimum bandwidth required for a PCM system with a sampling rate of $f_s = 8 \text{ kHz}$ and $N = 256$ quantization levels, we can use the formula:

    \begin{equation}
    BW = 2B = 2 \cdot (1 + \frac{79688}{10^6}) \cdot R
    \end{equation}
    
    where $B$ is the bandwidth, $\alpha$ is the excess bandwidth factor, and $R$ is the bit rate.
    
    The bit rate for a PCM system can be calculated as:
    
    \begin{equation}
    R = \log_2(N) \cdot f_s
    \end{equation}
    
    Substituting the given values, we get:
    
    \begin{equation}
    R = \log_2(256) \cdot 8 \text{ kHz} = 16 \text{ kbps}
    \end{equation}
    
    Therefore, the minimum bandwidth required for the telephone system using PCM is:
    
    \begin{equation}
    BW = 2B = 2 \cdot (1 + \frac{79688}{10^6}) \cdot 16 \text{ kbps} = 35200 \text{ Hz}
    \end{equation}
    
    So the minimum bandwidth needed is $35200 \text{ Hz}$.

\end{answer_box*}

\question
{
    find $t_0$ such that $sinc(2w(t - t_0 ))$ $<$ 0.001 for t $<$ 0
}
\begin{answer_box*}

\begin{align}
    \textbf{Case 1: t $<$ $t_0$}
\end{align}
    In this case, sinc(2w(t - $t_0$)) is given by

    $$sinc(2w(t - t_0)) = \frac{sin(2w(t - t_0))}{2w(t - t_0)}$$

    Since sin(x) $<$= x for all x, we have

    $$\frac{|sin(2w(t - t_0))|}{2w(t - t_0)} <= \frac{2w(t - t_0)}{2w(t - t_0)} = 1$$

    Therefore,

    $$|sinc(2w(t - t_0))| <= 1$$

    for all t $<$ $t_0$.

\begin{align}
    \textbf{Case 2: t = $t_0$}
\end{align}
    In this case, sinc(2w(t - $t_0$)) is given by

    $$sinc(2w(t - t_0)) = \frac{sin(2w(t - t_0))}{2w(t - t_0)} = \frac{sin(0)}{0} = \text{undefined}$$

    However, we can define sinc(0) to be 1, so we have

    $$sinc(2w(t - t_0)) = 1$$

    for t = $t_0$.

    Therefore, we have

    $$|sinc(2w(t - t_0))| <= 1$$

    for all t.
    
    To ensure that |sinc(2w(t - $t_0$))| $<$ 0.001 for t $<$ 0, we need to choose $t_0$ such that

    $$t_0 > \frac{\pi}{2w}$$

    This is because |sinc(x)| is maximized at x = 0, and |sinc(x)| $<$ 1 for all other x.

    Therefore, the smallest possible value of $t_0$ that satisfies both conditions is

    $$t_0 = \frac{\pi}{2w}$$
\end{answer_box*}

% ========================== end Question of Note 1=============================
% --------------------------------------------------------------------------------


% --------------------------------------------------------------------------------
% ==========================begin Question of Note 2==============================
\section{Note 2}
\question
{
    $P_RC (f + R_b ) + P_RC (f ) + P_RC (f - R_b ) = T_b \forall f \subset -R_b /2, R_b /2$
}

\begin{answer_box*}
    
Verification:

Step 1: Simplify the left-hand side

Using the definition of the raised cosine function, we can write each term on the left-hand side as a sum of two sinc functions:

\begin{equation}
    P_RC (f + R_b ) = \frac{1}{2}\left[\text{sinc}\left(\frac{f + R_b }{2Rc}\right) + \text{sinc}\left(\frac{f + R_b }{2Rc} - 1\right)\right]
\end{equation}

\begin{equation}
    P_RC (f ) = \frac{1}{2}\left[\text{sinc}\left(\frac{f }{2Rc}\right) + \text{sinc}\left(\frac{f }{2Rc} - 1\right)\right]
\end{equation}


\begin{equation}
    P_RC (f - R_b ) = \frac{1}{2}\left[\text{sinc}\left(\frac{f - R_b }{2Rc}\right) + \text{sinc}\left(\frac{f - R_b }{2Rc} - 1\right)\right]
\end{equation}


Substituting these expressions into the left-hand side, we get:

        $$P_RC (f +R_b ) + P_RC (f ) + P_RC (f -R_b ) =$$
        $$\frac{1}{2}\text{sinc}\left(\frac{f +R_b }{2Rc}\right) + \text{sinc}\left(\frac{f +R_b }{2Rc} - 1\right) + $$ 
        $$\text{sinc}\left(\frac{f }{2Rc}\right) + \text{sinc}\left(\frac{f }{2Rc} - 1\right) + $$
        $$\text{sinc}\left(\frac{f - R_b }{2Rc}\right) + \text{sinc}\left(\frac{f - R_b }{2Rc} - 1\right)    $$

Simplifying further, we get:

\begin{equation}
    P_RC (f +R_b ) + P_RC (f ) + P_RC (f -R_b ) = 
    \text{sinc}\left(\frac{f }{2Rc}\right) + \left[\text{sinc}\left(\frac{f +R_b }{2Rc}\right) + \text{sinc}\left(\frac{f -R_b }{2Rc}\right)\right]
\end{equation}


Step 2: Substitute into the original equation

Substituting the simplified expression for the left-hand side into the original equation, we get:

    $$\left[\text{sinc}\left(\frac{f }{2Rc}\right) + \left[\text{sinc}\left(\frac{f +R_b }{2Rc}\right) + \text{sinc}\left(\frac{f -R_b }{2Rc}\right)\right]\right] $$    
    $$\left[\text{sinc}\left(\frac{f +R_b }{2Rc}\right) + \text{sinc}\left(\frac{f -R_b }{2Rc}\right)\right] + $$
    $$\left[\text{sinc}\left(\frac{f + R_b }{2Rc}\right) + \text{sinc}\left(\frac{f - R_b }{2*Rc}\right) - T_b \right] = 0$$

Step 3: Simplify

Simplifying further, we get:
\begin{equation}
    \text{sinc}\left(\frac{f }{2Rc}\right) + \text{sinc}\left(\frac{f + R_b }{2Rc}\right) + \text{sinc}\left(\frac{f - R_b }{2*Rc}\right) = \frac{T_b }{2}
\end{equation}

Step 4: Check for all f in the range $-R_b /2, R_b /2$

We need to check if this equation holds for all f in the range $-R_b /2, R_b /2$, We can start by considering the case when $f = 0:$

\begin{equation}
    \text{sinc}(0) + \text{sinc}\left(\frac{R_b }{2Rc}\right) + \text{sinc}\left(-\frac{R_b }{2Rc}\right) = \frac{T_b }{2}
\end{equation}


Since the sinc function is symmetric around 0, we have sinc(-x) = sinc(x), which simplifies the equation to:
\begin{equation}
    2\text{sinc}(0) + \text{sinc}\left(\frac{R_b }{2Rc}\right) = \frac{T_b }{2}
\end{equation}

Since sinc(0) = 1, we can simplify further to:

\begin{equation}
    \text{sinc}\left(\frac{R_b }{2Rc}\right) = \frac{T_b }{2} - 2
\end{equation}

Now we need to check if this equation holds for all values of $R_b /(2Rc)$ in the range [-1, 1]. We can see that if $R_b /(2Rc)$ = 1, then the left-hand side of the equation becomes infinite, which means the equation does not hold. Therefore, the given condition does not hold for all f in the range $-R_b /2, R_b /2$.

\end{answer_box*}

\question
{
    $P_RC (f )$ exhibits odd symmetry with respect to $f = \frac{1}{2T_{b}}$. Verify odd symmetry.
}
\begin{answer_box*}

    Verification:

    To verify odd symmetry, we need to check whether the function satisfies the following property:
    \begin{equation}
        P_{RC}(f) = -P_{RC}(-f) \label{eq:odd_symmetry_condition}
    \end{equation}
    By Substituting in both:

    \begin{equation}
        P_{RC}(f) =  \frac{1}{T_b} \cdot \frac{\cos^2(-\pi f T_b)}{1 - \left({2\alpha (f)}{T_b}\right)^2} \label{odd_symmetry_eq1}  \\
    \end{equation}
    \begin{equation}
        -P_{RC}(-f) = - \frac{1}{T_b} \cdot \frac{\cos^2(-\pi f T_b)}{1 - \left({2\alpha (-f)}{T_b}\right)^2} \\
    \end{equation}  
    \begin{equation}
        -P_{RC}(-f) = - \frac{1}{T_b} \cdot \frac{\cos^2(-\pi f T_b)}{1 - \left({2\alpha f}{T_b}\right)^2} \label{odd_symmetry_eq2}
    \end{equation}
    Therefore, we can conclude by comparing equation1 \ref{odd_symmetry_eq1} to equation2 \ref{odd_symmetry_eq2} that $P_{RC}(f)$ exhibits odd symmetry with respect to $f = \frac{1}{T_b}$, as equation \ref{eq:odd_symmetry_condition} is satisfied.
\end{answer_box*}

\question
{
    We can recover ideal Nyquist pulse by plugging $\alpha  = 0$ to the RC pulse definition verify formally?
}
\begin{answer_box*}

    The raised cosine (RC) pulse is defined as:

    \begin{equation}
        P_{RC}(t) = \frac{1}{T_{b}} \cdot \frac{\cos\left(\pi\alpha t/T_{b}\right)}{\sin\left(\pi t/T_{b}\right)} \cdot \left(1-4\alpha^{2} \frac{t^{2}}{T^{2}}\right)
    \end{equation}

    To recover the ideal Nyquist pulse (Nyq), we need to set $\alpha = 0$ in this definition:

    \begin{equation}
        P_{Nyq}(t) = \frac{1}{T_{b}} \cdot \frac{\sin\left(\pi t/T_{b}\right)}{\pi t/T_{b}}
    \end{equation}

    Now, we need to verify whether the RC pulse with $\alpha = 0$ is equal to the ideal Nyquist pulse. Simplifying the expression, we have:

    \begin{equation}
        P_{RC}(t) \vert \alpha =0 = \frac{1}{T_{b}} \cdot \frac{1}{\sin\left(\pi t/T_{b}\right)}
    \end{equation}

    Comparing this with the definition of the ideal Nyquist pulse, we can see that it is indeed the same. Therefore, we can conclude that by setting $\alpha = 0$ in the definition of the raised cosine pulse, we can recover the ideal Nyquist pulse.
\end{answer_box*}

\question
{
    Prove the $P_RC$ in time domain?
}
\begin{answer_box*}
\end{answer_box*}

\question
{
    find $t_0$ such that $P_RC (t - t_0 ) < 0.001\ for\ t < 0$ ?
}
\begin{answer_box*}
    Since we are interested in finding $t_0$ such that $P_{RC}(t-t_0) < 0.001$ for $t<0$, we can restrict our attention to the range $t\in[-t_0,0]$. Therefore, we need to find $t_0$ such that:

\begin{equation}
    P_{RC}(t-t_0) < 0.001, \quad \text{for } -t_0 \leq t \leq 0
\end{equation}

Using the expression for $P_{RC}(t-t_0)$, we can rewrite the above inequality as:

\begin{equation}
    1 + \cos\left(\frac{\pi(t-t_0)}{T_b}\right) < 0.001, \quad \text{for } -t_0 \leq t \leq 0
\end{equation}

Since $\cos(\theta) \leq 1$ for all $\theta$, we can simplify the above inequality to:

\begin{equation}
    \frac{1}{T_b} < 0.001, \quad \text{for } -t_0 \leq t \leq 0
\end{equation}

which implies that

\begin{equation}
    t_0 > 1000T_b.
\end{equation}

Therefore, if we shift the raised cosine pulse to the right by $t_0 > 1000T_b$, then the pulse will satisfy $P_{RC}(t-t_0) < 0.001$ for $t<0$.
\end{answer_box*}

\question
{
    verify orthogonality of SRRC ?
}
\begin{answer_box*}
We aim to prove that the orthogonality of SRRC pulses is given by the fact that the integral of their product over the real line is zero for all non-zero $n$:


\begin{equation}
    \int_{-\infty}^{\infty} g_T(t)g_T(t-nT_b)dt = 0 \quad \text{for all } n \neq 0
\end{equation}


Substituting $g_T(t)$ and $g_T(t-nT_b)$ into the integral, we get:


\begin{equation}
\int_{-\infty}^{\infty} g_T(t)g_T(t-nT_b)dt = \int_{-\infty}^{\infty} \sin\left(\frac{\pi t}{T_b}\right)\cos\left(\frac{\pi \alpha t}{T_b}\right)\sin\left(\frac{\pi(t-nT_b)}{T_b}\right)\cos\left(\frac{\pi \alpha(t-nT_b)}{T_b}\right)dt
\end{equation}

$$=\frac{1}{4}\int_{-\infty}^{\infty} \cos\left(\frac{\pi(t-u)}{T_b}\right)-\cos\left(\frac{\pi(t+u)}{T_b}\right)du$$
$$\times\left[\left(\frac{\pi(t-u)}{T_b}\right)^2-\pi^2\alpha^2\right]\left[\left(\frac{2\alpha(t-u)}{T_b}\right)^2-1\right]+\frac{1}{4}\int_{-\infty}^{\infty} \cos\left(\frac{\pi(t-u)}{T_b}\right)+\cos\left(\frac{\pi(t+u)}{T_b}\right)du$$
$$\times\left[\left(\frac{\pi(t-u)}{T_b}\right)^2-\pi^2\alpha^2\right]\left[\left(\frac{2\alpha(t+u)}{T_b}\right)^2-1\right]$$

Substituting $u=t-nT_b$, we get:

$$\int_{-\infty}^{\infty} g_T(t)g_T(t-nT_b)dt = \frac{1}{4}\int_{-\infty}^{\infty} \cos\left(\frac{\pi u}{T_b}\right)-\cos\left(\frac{\pi(u+2nT_b)}{T_b}\right)du$$
$$\times\left[\left(\frac{\pi u}{T_b}\right)^2-\pi^2\alpha^2\right]\left[\left(\frac{2\alpha u}{T_b}\right)^2-1\right]+\frac{1}{4}\int_{-\infty}^{\infty} \cos\left(\frac{\pi u}{T_b}\right)+\cos\left(\frac{\pi(u+2nT_b)}{T_b}\right)du$$
$$\times\left[\left(\frac{\pi u}{T_b}\right)^2-\pi^2\alpha^2\right]\left[\left(\frac{2\alpha(u+2nT_b)}{T_b}\right)^2-1\right]$$

The first integral is zero because the integrand is an odd function of $u$. The second integral is also zero because the integrand is an even function of $u$ and the limits of integration are symmetric around $u=0$.

Therefore, we have shown that the integral of the product of two SRRC pulses, $g_T(t)$ and $g_T(t-nT_b)$, over the entire real line is zero for all $n\neq 0$. This proves that SRRC pulses are orthogonal to each other.
\end{answer_box*}

\question
{
    If $g_T (t)$ is delayed by multiples of $T_b$ , then the area of multiplication = 0. Why is this important for detection?
}
\begin{answer_box*}
\begin{itemize}
    \item The property that the area of multiplication is zero when the filter impulse response $g_T (t)$ is delayed by multiples of the symbol period $T_b$ is crucial for detection because it prevents inter-symbol interference (ISI).

    \item ISI occurs when symbols from adjacent time intervals overlap in time, causing distortion in the received signal. By ensuring that the area of multiplication is zero for delays that are multiples of T b , we effectively eliminate ISI.
    
    \item This is achieved by shifting the filter to a different time interval where it has no overlap with adjacent symbols. As a result, each symbol is sampled at a time when it has no interference from neighboring symbols, making it easier to detect correctly.
\end{itemize}
\end{answer_box*}

\question
{
    What is the maximum bit rate for any pulse?
}
\begin{answer_box*}
    $R_b = 2W$
\end{answer_box*}

\question
{
    Plot T.D \& F.D pulse using MATLAB
}
\begin{answer_box*}
\end{answer_box*}
\question
{
    How do we know that there wont be any uncontrolled ISI in addition to ISI from $a_{k-1}$ ?
}
\begin{answer_box*}
    \begin{itemize}
        \item To prevent uncontrolled inter-symbol interference (ISI) beyond that caused by the previous symbol (k-1), the pulse shape and system bandwidth must be carefully designed.

        \item The pulse shape should be selected to eliminate ISI at the decision points and minimize energy outside the decision window. This ensures that ISI from adjacent symbols is negligible.
        
        \item Therefore, the system bandwidth should be limited to prevent frequency-dependent attenuation of the transmission medium from causing ISI. This ensures that symbols are transmitted with minimal distortion and that ISI from distant symbols is also minimized.
    \end{itemize}
\end{answer_box*}
\question
{
    How $\hat{a}_k = x_k - \hat{a}_{k-1}$ is equivalent to :    
\begin{equation}
    \hat{a}_k =
    \begin{cases}
        -1             &\text{if } x_k = -2 \\
        1              &\text{if } x_k =  2 \\
        -\hat{a}_{k-1} &\text{if } x_k =  0
    \end{cases}
\end{equation}

    
}
\begin{answer_box*}
\begin{itemize}
    \item  When $x_k = -2$: Substituting $x_k = -2$ into the equation gives $\hat{a}_k = -2 - \hat{a}_{k-1}$, from the definition of $\hat{a}_k$ in the original equation, we have $\hat{a}_k = -1$, so substituting this value gives $\hat{a}_{k-1} = -2 - (-1) = -1$. Therefore, we have $\hat{a}_k = -1$ for $x_k = -2$.
    \item  When $x_k = 2$: Substituting $x_k = 2$ into the equation gives $\hat{a}_k = 2 - \hat{a}_{k-1}$, from the definition of $\hat{a}_k$ in the original equation, we have $\hat{a}_k = 1$, so substituting this value gives $\hat{a}_{k-1} = 2 - 1 = 1$. Therefore, we have $\hat{a}_k = 1$ for $x_k = 2$.
    \item When $x_k = 0$: Substituting $x_k = 0$ into the equation gives $\hat{a}_k = 0 - \hat{a}_{k-1}$, which can be rewritten as $\hat{a}_k = -\hat{a}_{k-1}$. Therefore, we have $\hat{a}_k = -\hat{a}_{k-1}$ for $x_k = 0$.  
\end{itemize}
\end{answer_box*}
\question
{
    Would detection differ if $w_0 = 1$?
}
\begin{answer_box*}
    No, since the duobinary sequence will be inverted but detection is the same of binary input data
\end{answer_box*}

\question
{
    Another version of correlative encoding is called modified duobinary, where
    \begin{equation}
        \begin{cases}
        P (n {T}_b) &= -1 \quad \text{if } n = 0 \quad \text{Repeat the analysis \& find P (t), P (f ). Explain why it is preferrable} \\
        P (n {T}_b) &= 1 \quad \text{if } n = 0 \\
        P (n {T}_b) &= 0 \quad \text{otherwise} 
        \end{cases}
    \end{equation}
    in channels that doesnt pass DC levels.
}
\begin{answer_box*}

    Modified duobinary encoding is a type of correlative encoding that modifies the duobinary encoding scheme. 
    Specifically, it changes the polarity of the -1 symbol.\\
    The autocorrelation function of the modified duobinary signal is given by:
\begin{equation}
    R(\tau ) = \int_{-\infty}^{\infty} P (t)P (t - \tau )dt
\end{equation}

where,

\begin{equation}
    R(\tau ) = \int_{-\infty}^{\infty} P (t)P (t)dt + \int_{-\infty}^{\infty} P (t)P (t - \tau )dt \\
     = T_b + (-1)^{\frac{\tau}{T_b}}T_b \\
\end{equation}

\begin{equation}
    \begin{cases}
        \frac{1}{2} & \text{if } t = 0 \\
        \frac{1}{4} & \text{if } t = \pm T_b \\
        0 & \text{otherwise.}
    \end{cases}
\end{equation}

\begin{equation}
    F (f ) = \left(\frac{1}{2} + \frac{3}{4} e^{-j2\pi f T b } + \frac{1}{4} e^{j2\pi f T b}\right).
\end{equation}

\begin{equation}
    S(f) = |F(f)|^2 = \frac{1}{2} \left(1 + \cos(2\pi fT_b)\right)
\end{equation}

By comparing this with the PSD of the standard duobinary signal, we can see that the modified duobinary signal has a wider bandwidth. This is due to the presence of a DC component in the modified duobinary signal.
Therefore, the modified duobinary signal is preferable in channels that do not pass DC levels. This is because the wider bandwidth of the modified duobinary signal will allow it to pass through channels that do not pass DC levels more easily.

\end{answer_box*}
% ========================== end Question of Note 2===============================
% --------------------------------------------------------------------------------
% ========================== start Question of Note 3===============================
% --------------------------------------------------------------------------------
\section{Note 3}
\question
{
}
\begin{answer_box*}
\end{answer_box*}

\question
{
}
\begin{answer_box*}
\end{answer_box*}



% ========================== end Question of Note 3===============================
% --------------------------------------------------------------------------------

% ==============================================================================
%
% Chapter 2
%
% ==============================================================================

\chapter{Introduction to information theory}
\noindent\makebox[\linewidth]{\rule{\textwidth}{0.4pt}}



% ==============================================================================
%
% Reference
%
% ==============================================================================

\chapter{References}
\noindent\makebox[\linewidth]{\rule{\textwidth}{0.4pt}}

\label{finalpage}
\end{document}